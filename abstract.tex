%\abstracttitle
% Single spacing can be turned on for the abstract
%
{\singlespacing

% NO MORE THAN 300!!!

Programming is difficult in a general sense. Tuning for parallel performance adds more complexity to a program. Complexity means more potential bugs and greater difficulty to introduce changes to existing code, reducing its maintainability. Coupling these observations with the fact that most programs today are designed for a particular piece of hardware means that current applications have a limited lifespan. 

As hardware continues to evolve, new and different products will continue to emerge. The current trends in computer architectures dictate that future hardware configurations will contain different kinds of chips, mixing their individual performance capabilities, both for maximising performance for existing programs, as well as for widening the range of applications that can be created to tackle problems where high performance is a decisive factor.

However, creating programs for these heterogeneous architectures with current programming models represents a formidable task. New performance portability techniques are needed. While custom versions of programs for specific architectures might always yield better results than programs made with these new techniques, the savings in terms of human effort are undeniable and could mean a much more viable application development life-cycle, with longer intervals between high performing versions.

The aim of the project is to explore cost-effective (in terms of time and human effort) ways to exploit parallelism targeting heterogeneous architectures with the same code base. In this report a case for the portability of performance is made and a research methodology is consolidated to evaluate approaches to performance portability.

}
